\chapterConclusion

Заміри швидкодії та результати роботи згорткових нейромереж та алгоритму Б-К показують
їхню доцільність для прибирання викладача та їх використання на смартфоні. 
Можна стверджувати, що згорткова нейромережа YOLOv5 є оптимальним способом
прибирання викладача з відео. Також варто відмітити, що алгоритм Б-К
працює краще за протестовані нейромережі,
коли викладач має якийсь предмет в руках. 


Інформаційна технологія створення панорамних слайдів дає досить якісні результати, 
однак, чим менша панорама ~---~ тим менше часу на одну ітерацію її створення. Дану проблему 
можна вирішити детектуванням дошки, тоді панорамний знімок буде менший. 

Швидка медіана справді швидша за звичайну темпоральну, тому це також
прискорило роботу системи. Результати її застосування свідчать про підвищення
якості панорамних знімків.
