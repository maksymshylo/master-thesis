\chapterConclusion

Заміри швидкодії та результати роботи згорткових нейромереж та алгоритму Б-К показують
їхню доцільність для прибирання викладача та їх використання на смартфоні. 
Можна стверджувати, що згорткова нейромережа YOLOv5 є оптимальним способом
прибирання викладача з відео. Також варто відмітити, нейромережі 
працюють гірше для задачі видалення викладача, коли в нього є щось в руках, 
оскільки вони знаходять лише людину, в той час як Б-К алгоритм ~---~ всі рухомі об'єкти.


Інформаційна технологія створення панорамних слайдів дає досить якісні результати, 
однак, чим більша панорама ~---~ тим більше часу на одну ітерацію її створення. Дану проблему можна вирішити 
детектуванням дошки, тоді панорамний знімок буде менший. 

Результати застосування швидкої медіани відповідають припущенню, висунутому в минулому розділі.
Її використання прискорило роботу інформаційної технології. 
