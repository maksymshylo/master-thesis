\section{Пропоновані рішення}


У даній статті описується алгоритм напівавтоматичного 
створення слайдів на основі відеозапису лекції [1].
За допомогою наведеного алгоритму можна обробляти відео 
у режимі реального часу, адже кожна його ітерація потребує 
лише поточний на попередні кадри. Між кожною парою 
сусідніх кадрів відео розраховується бінарна маска областей, де 
потенційно знаходяться рухомі об’єкти. 


Ті частини зображення, де не було помічено рух, зберігаються до 
зображення, яке ми в даній роботі називаємо панорамою, тому що 
це зображення може розширюватись у випадку, коли камера рухається 
навмисно або хитається через небажаний вплив на неї. 
Отриману панораму ми порівнюємо з тими панорамами, що було отримано на 
попередніх кроках, і створюємо новий слайд, якщо було помічено 
достатню кількість змін.


Інформаційна технологія, що реалізує даний алгоритм, працює в 
напівавтоматичному режимі – вона потребує від користувача введення 
деяких параметрів. 
Ці параметри визначають крок, з яким треба обирати кадри для визначення руху, 
ступінь згладжування маски рухомих об’єктів під час її пошуку, крок 
між панорамами для перевірки необхідності створення нового 
слайду та кількість необхідних змін між двома панорамами для створення нового слайду.



\usetikzlibrary{arrows,positioning,shapes}
\begin{figure}
\begin{center}
\begin{tikzpicture}[node distance=4mm, >=latex',
 block/.style = {draw, rectangle, minimum height=10mm, minimum width=28mm,align=center},
rblock/.style = {draw, rectangle, rounded corners=0.5em},
tblock/.style = {draw, trapezium, minimum height=10mm, 
                 trapezium left angle=75, trapezium right angle=105, align=center},
                        ]
    \node [rblock]                           (video)        {Відео};
    \node [block, below=of video]            (mov_objects)  {Побудова маски\\ рухомих об’єктів};
    \node [block, right=of mov_objects]      (pan)          {Побудова панорамами};
    \node [block, right=of pan]              (denoise)      {Зниження рівня\\ шуму};
    \node [rblock,above=of denoise]          (slides)       {Слайди};

%% paths (borowed from Harish Kumar)
    \path[draw,->] (video)         edge    (mov_objects)
                   (mov_objects)   edge    (pan)
                   (pan)           edge    (denoise)
                   (denoise)       edge    (slides)
                   ;
\end{tikzpicture}
\end{center}
\end{figure}