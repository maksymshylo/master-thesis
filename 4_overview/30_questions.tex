\section{Питання щодо постановки задачі}

Перше фундаментальне питання полягає в тому,
навіщо потрібно мінімізувати саме ті функції, що було вказано вище.
В роботі \cite{blanz:romdhani:vetter} вибір функції витрат
\eqref{eq:energy:blanz} мотивується оцінкою апостеріорного максимуму.
Проте функція \eqref{eq:energy:face2face} не має підґрунтя окрім того,
що її мінімізація дає блискучі результати, проте не ідеальні
(рис. \ref{fig:questions:trump}).
\begin{figure}[h]
  \centering
  \begin{subfigure}[b]{0.4\textwidth}
    \centering
    \includegraphics[width=\textwidth]{images/trump}
    \caption{Фрагмент оригінального фото Дональда Трампа}
  \end{subfigure}
  \begin{subfigure}[b]{0.4\textwidth}
    \centering
    \includegraphics[width=\textwidth]{images/trump-reconstructed}
    \caption{Реконструкція поверхні обличчя Дональда Трампа}
  \end{subfigure}
  \caption{Некоректно реконструйований ніс}
  \label{fig:questions:trump}
\end{figure}

У формулі \eqref{eq:energy:face2face} є
вагові коефіцієнти $\omega_c$, $\omega_r$ і $\omega_l$,
походження яких не пояснюється;
невідомо, чому значення мають бути сами такими,
і чи вони залежать від вхідних даних.

У статтях зазначається,
що енергія є складно влаштованою функцією,
яка має багато локальних мінімумів.
Для подолання цієї складності
автори згадуваних статей пропонують
модифікації класичних алгоритмів
та цільової функції.
Досліджень цих локальних екстремумів не було знайдено:
за якими параметрами, за яких умов і наскільки глибокі локальні мінімуми ---
можливо, ця інформація допомогла б знайти рішення,
яке дасть змогу відрізнити локальний мінімум від глобального.
