\section{Постановка задачі}

Спільною рисою методів, що дають добрі результати,
є використання породжувальної моделі обличчя
для виконання аналізу через синтез.
Метою є мінімізація функції, що називається енергією.
В першій роботі Бланца и Феттера \cite{blanz:romdhani:vetter}
енергія складалася з
суми квадратів різниць кольорів пікселів
оригінального та синтезованого зображення,
суми квадратів параметрів форми моделі обличчя,
що мають стандартний нормальний розподіл,
та суми квадратів відхилень параметрів камери від початкових значень,
що визначалися людиною-оператором:
\begin{equation}\label{eq:energy:blanz}
  E_N\left( x, \theta \right)
  = \sum_{i \in I' \subset I}
      \frac{\left\| y_i \right\|^2}{\sigma^2_N}
  + \sum_{j = 1}^{n} x_j^2
  + \sum_{k = 1}^{m}
      \frac{\left\| \theta^m - \hat{\theta}^m \right\|^2}
           {\sigma^2_{\theta^m}}.
\end{equation}
Дисперсія $\sigma_N^2$ відповідає шуму на оригінальному зображенні.
Зауважимо, що беруться не всі пікселі зображення,
а тільки ті, що відповідають пікселям, де намальовано модель.
Щоб кількість доданків завжди була однаковою,
на кожній ітерації алгоритму випадковим чином
обирається певна кількість трикутників
з ймовірністю, що пропорційна до їх площі.

Процедура мала наступний вигляд:
\begin{enumerate}
  \item на перших кроках використовується модель
    з меншою кількістю трикутників та вершин;
  \item спочатку використовуються лише параметри камери
    та найзначущі параметри моделі (перші головні компоненти),
    потім поступово додаються нові компоненти;
  \item спочатку дисперсія $\sigma_N^2$ обирається достатньо великою,
    аби більше звертати увагу на грубіші відхилення,
    а з кожною ітерацією її зменшують,
    щоб отримувати все більш схожу модель.
\end{enumerate}

У новітніх роботах для кращих результатів реконструкції використовуються
додаткові відомості про обличчя~---~опорні точки.
Знаходження особливих точок на зображенні голови
являє собою окрему складну задачу.
Одним з популярніших на даний час є метод, опублікований в статті
``Вирівнювання обличчя за одну мілісекунду'',
що використовується в бібліотеці з відкритим кодом dlib \cite{Kazemi:2014}.
Результати, що дає алгоритм, можна побачити на рис. \ref{fig:problems:dlib}.
\begin{figure}[h]
  \centering
    \includegraphics[width=0.9\textwidth]{images/dlib}
  \caption{Приклади знаходження опорних точок обличчя}
  \label{fig:problems:dlib}
\end{figure}

Також популярним є евристичний підхід,
який полягає у використанні зміщеної оцінки дисперсії
замість суми квадратів відхилень.
Це дозволяє брати різну кількість точок на кожній ітерації.
Нова цільова функція має вигляд
\begin{equation}\label{eq:energy:face2face}
  E\left( x, \theta \right)
  = \omega_c \cdot \sum_{i \in I' \subset I} \frac{y_i^2}{\left| I' \right|}
  + \omega_r \cdot \sum_{j = 1}^{n} x_j^2
  + \omega_l \cdot \sum_{l \in L} \frac{\Delta_l^2\left( x \right)}
                                       {\left| L \right|},
\end{equation}
де $\Delta_l^2$~---~функція,
яка перетворює параметри моделі
на квадрат відхилення її опорних точок від реальних,
а коефіцієнти $\omega_c$, $\omega_r$ і $\omega_l$
обираються рівними $1$, $2.5 \cdot 10^{-5}$ і $10$ відповідно.
