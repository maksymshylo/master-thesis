\chapterConclusion

Проведено огляд існуючих алгоритмів реконструкції
з теоретичної точки зору,
більш детальне дослідження яких наведено у наступних розділах дисертації.
Запропоновані методи досить складно порівняти з практичної точки зору:
для цього треба реалізувати їх на одній мові програмування
та перевірити на одній тестовій виборці на однакових обчислювальних машинах.
Проте було вказано математичні неточності,
що були знайдені при вивченні даних методів.
В результаті аналізу стало зрозуміло,
що на даний момент
для просторової реконструкції людського обличчя за одним зображенням
доцільно використовувати генеративну модель обличчя.
Цей факт використано в наступному розділі для постановки задачі розпізнавання.
