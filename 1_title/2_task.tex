\begin{center}
    {\textbf{НАЦІОНАЛЬНИЙ ТЕХНІЧНИЙ УНІВЕРСИТЕТ УКРАЇНИ}}\\[-0.5ex]
    {\textbf{``КИЇВСЬКИЙ ПОЛІТЕХНІЧНИЙ ІНСТИТУТ\\ імені ІГОРЯ СІКОРСЬКОГО''}}\\[-0.5ex]
    {\textbf{\faculty}}\\
    {\textbf{\department}}
\end{center}
\begin{flushleft}
    Рівень вищої освіти – другий (магістерський) \\
    Спеціальність – \specialityCode~\specialityTitle \\
    Освітньо-наукова програма: «\specialityName»
\end{flushleft}
\begin{adjustwidth}{21em}{1em}
    \begin{flushright}
        \MakeUppercase{затверджую}\\
        Завідувач кафедри\\
        $\underset{\text{\textit{\tiny(підпис)}}}
            {\text{\underline{\phantom{(підпис)}}}}$
        $\underset{\text{\textit{\tiny(ініціали, прізвище)}}}
            {\text{\underline{\departmentHead}}}$\\
        ``${\text{\underline{\hspace{2em}}}}$''
        ${\text{\underline{\hspace{6em}}}}$
        \passYear~р.
    \end{flushright}
\end{adjustwidth}
\begin{center}
    \MakeUppercase{\textbf{завдання}} \\
    \textbf{на магістерську дисертацію студенту} \\
    Шилу Максиму Костянтиновичу
\end{center}
\begin{enumerate}
    \item[1.]
        Тема дисертації «\theme» \\
        науковий керівник дисертації \mentorName, \\
        кандидат технічних наук, \\
        затверджені наказом по університету від
        «${\text{\underline{\hspace{2em}}}}$»
        ${\text{\underline{\hspace{6em}}}}$  \passYear р. №
    \item[2.]
        Термін подання студентом дисертації ${\text{\underline{\hspace{10em}}}}$
    \item[3.]
        Об’єкт дослідження~---~відеозаписи лекцій.
    \item[4.]
        Предмет дослідження~---~автоматична обробка
        відеозаписів.
    \item[5.]
        Перелік завдань, які потрібно розробити:
        \begin{enumerate}
            \item
                  вивчити та доповнити математичні методи,
                  що допоможуть визначати рухомі об'єкти;
            \item
                  опрацювати літературу про згорткові нейронні мережі, які застосовуються
                  для знаходження в кадрі людини;
            \item
                  вивчити методи обробки та порівняння зображень для створення панорамних знімків;
            \item
                  розробити демонстраційне програмне забезпечення.
        \end{enumerate}
    \item[6.]
        Орієнтовний перелік ілюстративного матеріалу: 15 слайдів.
    \item[7.]
        Орієнтовний перелік публікацій: 2 публікації.
    \item[8.]
        Дата видачі завдання ${\text{\underline{\hspace{10em}}}}$
\end{enumerate}

\begin{center}
    Календарний план
\end{center}

\begin{table}[H]
    \centering
    \begin{tabular}{|p{0.2in} | p{2.9in} | p{2.1in} | p{0.9in}|}
        \hline
        № з/п & Назва етапів виконання магістерської дисертації            & Термін виконання етапів магістерської дисертації & Примітка \\
        \hline
        1.    & Постановка задачі та мети                                  & 21.09.2020 - 15.10.2020                          &          \\
        \hline
        2.    & Дослідження літератури                                     & 15.09.2020 - 10.01.2021                          &          \\
        \hline
        3.    & Створення математичної моделі створення панорамних слайдів & 10.01.2021 - 20.06.2021                          &          \\
        \hline
        4.    & Імплементація інформаційної технології                     & 20.06.2021 - 15.01.2022                          &          \\
        \hline
        5.    & Аналіз і висновки                                          & 15.01.2022 - 20.02.2022                          &          \\
        \hline
        6.    & Створення документації                                     & 20.03.2022 - 20.05.2022                          &          \\
        \hline
    \end{tabular}
\end{table}


\begin{table}[H]
    \centering
    \begin{tabular}{p{3.7in} p{1.2in} p{1.8in}}

        Студент                      & ${\text{\underline{\hspace{4em}}}}$  & М. К. Шило         \\
        Науковий керівник дисертації & ${\text{\underline{\hspace{4em}}}} $ & Є. В. Водолазський
    \end{tabular}
\end{table}

\clearpage

