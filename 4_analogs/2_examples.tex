\subsection{Перша робота по оцифровуванню дошки}
У 2004 році інженери з Microsoft Research Zhengyou Zhang
та Li-wei He представили свій алгоритм по скануванню написів
білої дошки \cite{zhang:2004}. Система оброблює фотографії білої дошки, локалізує область
написів, вирівнює у прямокутну форму дошку та бінаризує написи без втрати
кольору (рис. \ref{fig:zhang:2004}).

\begin{figure}[H]
  \centering
  \subfloat[До обробки]{
      \includegraphics[width=0.35\textwidth]{images/zhang_2004_1}
  }
  \subfloat[Після обробки]{
      \includegraphics[width=0.35\textwidth]{images/zhang_2004_2}
  }
  \caption{Демонстрація роботи алгоритму інженерів з Microsoft \cite{zhang:2004}
    \label{fig:zhang:2004}
  }
\end{figure}
Автори реалізували склейку зроблених з різних
ракурсів зображень дошки за допомогою гомографії.
Гарна якість оцифрування дошки
досягається насамперед тим, що вона має білий колір, що в свою чергу
накладає обмеження на використання технології з дошками відмінного
від білого кольорів.
У наступній свої роботі \cite{zhang:2007} ці ж самі автори побудували
технологію, яка в реальному часі оброблює відеозапис та видаляє людину біля
дошки за допомогою часової медіани, але тут немає
панорамного склеювання знімків. Головна ідея роботи полягала у розробці
програми для телеконференцій.

\subsection{Автоматичне сканування дошки}
Автори роботи \cite{wienecke} створили програму, яка переводить написи на білій
дошці у цифрові (рис. \ref{fig:wienecke}). Вони реалізували локалізацію тексту та подальшу
його обробку. Дана технологія не вирішує проблему перекривання викладачем написів,
а також не дозволяє використовувати дошку, що має відмінний від білого колір.
\begin{figure}[H]
  \centering
  \subfloat[Детекція написів]{
      \includegraphics[width=0.35\textwidth]{images/wienecke_1}
  }
  \subfloat[Обробка написів]{
      \includegraphics[width=0.35\textwidth]{images/wienecke_2}
  }
  \caption{Демонстрація роботи сканування дошки \cite{wienecke}
    \label{fig:wienecke}
  }
\end{figure}
Можна помітити (рис. \ref{fig:wienecke}), що, як і в попередній роботі, гарна якість виокремлення написів
досягається тим, що дошка білого кольору.

\subsection{Відстежування об'єкту та віднімання фону}
У 2012 році науковці зі Стенфордського університету Alex Gonzalez,
Bongsoo Suh, Eun Soo Choi представили технологію \cite{suh} локалізації дошки
(навіть такої, яка розділена на частини), відстеження викладача та його
подальше прибирання. Алгоритм також може працювати з різними кольорами
дошок. Для прибирання викладача і всіх рухомих об'єктів автори також
використали часову медіану (рис. \ref{fig:suh}).

Дана програма не працює в реальному часі, оскільки всі операції над кадрами
відео займають тривалий час, а також саме відео перед обробкою
піддають компресії.
\begin{figure}[H]
  \centering
  \includegraphics[width=0.5\textwidth]{images/suh}
  \caption{Демонстрація роботи авторів зі Стенфордського університету \cite{suh}
    \label{fig:suh}
  }
\end{figure}
Головною особливістю даної роботи є те, що  алгоритм автоматично локалізує
різну кількість дошок. Однак, варто відмітити, що тестування відбувалось на
відео лекціях, де камера знімає всю дошку і не рухається за викладачем.

\subsection{Відокремлення написів дошки}
У 2014 році науковці з Тайванського університету представили свій алгоритм \cite{yeh}
оцифровування дошки (рис. \ref{fig:yeh}). Для видалення викладача, автори застосували
алгоритм кластеризації k-means. Для отримання бінаризованих написів з дошки
використане адаптивне вирівнювання. Варто відмітити гарну
якість власного методу зменшення шуму.
\begin{figure}[H]
  \centering
  \subfloat[До обробки]{
      \includegraphics[width=0.45\textwidth]{images/yeh_1}
  }\\
  \subfloat[Після обробки]{
      \includegraphics[width=0.45\textwidth]{images/yeh_2}
  }
  \caption{Демонстрація роботи сканування дошки \cite{yeh}
    \label{fig:yeh}
  }
\end{figure}
Автори не надали час обробки всього відео. Щоб отримати якісну сегментацію
дошки та викладача, потрібно, щоб кадр містив не просто викладача, а викладача з кольором одягу 
сильно відмінним від кольору дошки. Це потрібно для коректної роботи алгоритму k-means.
Тому немає гарантії, що якийсь рухомий об'єкт не буде класифікований як дошка під час класифікації.

\subsection{Сучасна робота}
Окремо зазначимо роботу \cite{davila:2017}
науковців з університету Рочестер. Автори Kenny Davila та Richard Zanibbi
використали просторово-часовий індекс для виокремлення написів і викладача (рис. \ref{fig:davila:2017}).
Відбувається видалення не самого викладача,  а його контурів після бінаризації картинки.
Варто відмітити, що і тут камера має бути нерухомою.
\begin{figure}[H]
  \centering
  \subfloat[До обробки]{
      \includegraphics[width=0.45\textwidth]{images/davila_2017_1}
  }\\
  \subfloat[Після обробки]{
      \includegraphics[width=0.45\textwidth]{images/davila_2017_2}
  }
  \caption{Демонстрація роботи сканування дошки \cite{davila:2017}
    \label{fig:davila:2017}
  }
\end{figure}
Пізніше, ці ж автори створили повністю згорткову нейронну мережу \cite{davila:2021}
для обробки написів дошки. Дана нейронна мережа LectureNet досить добре бінаризує написи з дошки
(рис. \ref{fig:davila:2021}).
\begin{figure}[H]
  \centering
  \includegraphics[width=0.7\textwidth]{images/davila_2021}
  \caption{Приклад роботи авторів з Рочестер \cite{davila:2021}} 
  \label{fig:davila:2021}
\end{figure}

\clearpage