\chapterConclusion

Як ми бачимо, всі вищеописані алгоритми є частковими (тобто не повністю вирішують всі проблеми
з оцифруванням дошки): якісь методи опрацьовують тільки написи з дошки, 
при чому досить непогано; якісь роблять панорамні знімки тільки з фотографій дошки.
Спільними елементами вищеописаних алгоритмів-аналогів є використання 
темпорального медіанного фільтру для видалення рухомих об'єктів.
У даній роботі даний метод теж має місце, але вже для знешумлення
вихідних слайдів, а не видалення лектора з відео.
Найбільшими недоліками всіх аналогів є час обробки відео. Більшість технологій
тестувались на відео поганої якості, наприклад 240р - 720р. 
Також жодний аналог не пропонує створювати панорамні знімки по мірі руху камери під
час зйомки.