\chapterConclusion

Як ми бачимо, всі вищеописані алгоритми не повністю вирішують поставлену нами задачу
оцифрування дошки: деякі методи обробляють тільки написи з дошки,
частина аналогів створює панорамні знімки тільки з фотографій дошки.
Спільними елементами вищеописаних аналогів є використання 
темпорального медіанного фільтру для видалення рухомих об'єктів.
У поточній роботі даний метод теж має місце, але вже для знешумлення
вихідних слайдів, а не для видалення лектора з відео.
Найбільшими недоліками всіх аналогів є час обробки відео.
Більшість технологій тестувалась на відео з низькою роздільною здатністю
від $352\times240$ до $1280\times720$,
хоча сучасні недорогі смартфони дозволяють записувати відео
розмірами $1920\times1080$, а більш дорогі моделі навіть $3840\times2160$,
що помітно позначиться на швидкодії існуючих алгоритмів.
Також жодний аналог не пропонує створювати панорамні знімки по мірі руху камери під
час зйомки.
