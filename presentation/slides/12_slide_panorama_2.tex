\begin{frame}
    \frametitle{Алгоритм створення панорами}
    \textbf{6.}
    Для визначення множини \(P_{W}^{i}\) нової панорами \(W^{i + s}\)
    рахуємо величини
    $
        x_{\min}^{i} = \min_{j = \overline{1,4}}\frac{( l_{j}^{i} )_{x}}{( l_{j}^{i} )_{z}},
        x_{\max}^{i} = \max_{j = \overline{1,4}}\frac{( l_{j}^{i} )_{x}}{( l_{j}^{i} )_{z}},
        y_{\min}^{i} = \min_{j = \overline{1,4}}\frac{( l_{j}^{i} )_{y}}{( l_{j}^{i} )_{z}},
        y_{\max}^{i} = \max_{j = \overline{1,4}}\frac{( l_{j}^{i} )_{y}}{( l_{j}^{i} )_{z}}.
    $ \\
    Позначимо
    $
        P_{W}^{i + s} =
        { 1,\ldots,\max( w^{i},x_{\max}^{i} - x_{\min}^{i} ) }
        \times
        { 1,\ldots,\max( h^{i},y_{\max}^{i} - y_{\min}^{i} ) }
    $ \\
    \textbf{Етап створення нової панорами.} \\
    \textbf{7.}
    Будуємо панораму $W^{i+s}$. Для зручності позначимо обернену матрицю $H^{'} = (H_{W}^{i})^{-1}$, числа
    $x_{\min}^{'} = \min(x_{\max}^{i}, 0), y_{\min}^{'} = \min(y_{\max}^{i}, 0)$ і відображення
    $
        f^i: p \rightarrow
        \left( \left| \frac{(H^{'} \cdot p)_x}{(H^{'} \cdot p)_y} \right|, \left| \frac{(H^{'} \cdot p)_y}{(H^{'} \cdot p)_z} \right| \right),
    $
    що перетворює координати з панорами до пікселів кадру за допомогою гомографії.
    Інтенсивність у пікселі $p$ панорами
    $W^{i+s}$ визначається за формулою
    \begin{equation*}
        W^{i + s}(p) =
        \begin{cases}
            F^{i + s}(f(p)),                            & f(p) \in P,                                     \\
            W^{i}( p + ( x_{\min}^{'},y_{\min}^{'} ) ), & p + ( x_{\min}^{'},y_{\min}^{'} ) \in P^{i},    \\
            0,                                          & p + ( x_{\min}^{'},y_{\min}^{'} ) \notin P^{i}.
        \end{cases}
    \end{equation*}
\end{frame}
