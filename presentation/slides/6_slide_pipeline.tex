\section{Процедура створення панорамних слайдів}
\begin{frame}
    \frametitle{Процедура створення панорамних слайдів}
    \usetikzlibrary{arrows,positioning,shapes}
    \begin{figure}[H]
        \begin{center}
            \begin{tikzpicture}[node distance=4mm, >=latex',
                    block/.style = {draw, rectangle, minimum height=10mm, minimum width=28mm,align=center},
                    rblock/.style = {draw, rectangle, rounded corners=0.5em},
                    tblock/.style = {draw, trapezium, minimum height=10mm,
                            trapezium left angle=75, trapezium right angle=105, align=center},
                ]
                \node [rblock]                           (video)        {Відео};
                \node [block, below=of video]            (mov_objects)  {Побудова маски\\
                    рухомих об'єктів або людини};
                \node [block, right=of mov_objects]      (pan)          {Побудова панорамами};
                \node [block, right=of pan]              (denoise)      {Зниження рівня\\ шуму};
                \node [rblock,above=of denoise]          (slides)       {Слайди};

                \path[draw,->] (video)         edge    (mov_objects)
                (mov_objects)   edge    (pan)
                (pan)           edge    (denoise)
                (denoise)       edge    (slides)
                ;
            \end{tikzpicture}
        \end{center}
    \end{figure}
\end{frame}
