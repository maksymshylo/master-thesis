\begin{frame}
    \frametitle{Вступ}
    \textit{Об'єкт дослідження}~---~відеозаписи лекцій. \\
    \textit{Предмет дослідження}~---~автоматична обробка відеозаписів. \\
    \textit{Метою роботи} є розробка алгоритму, що перетворює
    відеозапис лекції у панорамні знімки без викладача
    за умови, що камера може рухатись. \\
    Даний алгоритм має:
    \begin{itemize}
        \item враховувати дошки різного кольору;
        \item враховувати рух камери під час зйомки;
        \item давати мінімальну кількість дефектів на слайдах,
              таких як наявність фрагментів викладача;
        \item мати можливість ефективної реалізації на звичайному смартфоні.
    \end{itemize}
    Апробація результатів роботи:
    \begin{enumerate}
        \item Доповідь на
              \href{https://s3.eu-central-1.amazonaws.com/ucu.edu.ua/wp-content/uploads/sites/8/2021/04/Creating-Slides-from-Video-Lecture.pdf}{Masters Symposium on Advances in Data Mining,
                  Machine Learning, and Computer Vision в Українському Католицькому Університеті}.
        \item Стаття <<Перетворення відеозапису з дошки у слайд-шоу>> авторів
              Кригін В. М. та Шило М. К. прийнята до друку
              Міжнародним науковим журналом “Control Systems and Computers”
              (“Системи керування та комп'ютери”) та буде опублікована в № 2 (298) 2022 року.
    \end{enumerate}
\end{frame}
