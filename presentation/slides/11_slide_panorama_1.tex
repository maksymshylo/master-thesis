\section{Створення панорами}
\begin{frame}
  \frametitle{Алгоритм створення панорами}
  \textbf{Вхід:} два кадри \(F^{i}\) та \(F^{i + s}\), поточна панорама \(W^{i}\) (\(W^{1} = F^{1}\)). \\
  \textbf{Вихід:} панорама \(W^{i + s}\). \\
  \textbf{Етап отримання відповідних точок.} \\
  \textbf{1.}
  Знаходимо набір \(M^{i}\) пар відповідних пікселів між кадрами
  \(F^{i}\) і \(F^{i + s}\) і будуємо множину
  \(M^{'i} = \left\{ \left( p^{i},p^{i + s} \right) \in M^{i}:B_{p^{i}}^{i} = B_{p^{i + s}}^{i} = 0 \right\}\)
  тих пар відповідних пікселів, координати яких не належать області
  рухомих об'єктів.\\

  \textbf{2.}
  Якщо \(\left| {M'}^{i} \right| < 0.5 \cdot \left| M^{i} \right|\) або
  \(\left| {M'}^{i} \right| < 4\), завершуємо алгоритм з результатом
  \(W^{i + s} = W^{i}\).\\

  \textbf{3.}
  Знаходимо набір \(M_{W}^{i}\) пар відповідних пікселів між панорамою
  \(W^{i}\) і кадром \(F^{i + s}\) і будуємо множину
  \(M_{W}^{'i} = \left\{ \left( p_{W}^{i},p^{i + s} \right) \in M_{W}^{i}:\exists p^{i} \in P:\left( p^{i},p^{i + s} \right) \in M^{'i} \right\}\). \\

  \textbf{Етап обчислення матриці гомографії.}\\
  \textbf{4.}
  На базі множини \(M_{W}^{'i}\) пар відповідних точок знаходимо матрицю
  \(H_{W}^{i}\) гомографії, що співставляє пікселі кадру \(F^{i + s}\)
  та панорами \(W^{i}\). \\

  \textbf{Етап обчислення розміру нової панорами.} \\
  \textbf{5.}
  Рахуємо координати крайніх точок кадру \(F^{i + s}\) після застосування до них матриці
  \(H_{W}^{i}\).
  $l_{1}^{i} = H_{W}^{i} \cdot (0,0,1)^{T}, l_{2}^{i} = H_{W}^{i} \cdot (w - 1,0,1)^{T},  l_{3}^{i} = H_{W}^{i} \cdot (0,h - 1,1)^{T},l_{4}^{i} = H_{W}^{i} \cdot (h - 1,h - 1,1)^{T}$
 
\end{frame}
