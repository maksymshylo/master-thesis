\section{Висновки}
\begin{frame}
  \frametitle{Висновки}
  Побудовано алгоритм, що перетворює відеозапис лекції у панорамні знімки без викладача з камерою, що може
  рухатись. Створена інформаційна технологія по даному алгоритму. Програма є досить
  швидкою, оскільки може оброблювати відео з високою роздільною здатністю $1920 \times 1080$. 
  Наприклад обробка 45 хв відео (\url{https://youtu.be/_aGdtc23o1o}) на комп'ютері з 
  процесором Intel Core i5-7200U @ 4x 3,1GHz на операційній системі Ubuntu 21.04 займає
  всього 12 хвилин. Такі результати свідчать про можливу подальшу роботу над даною темою.

  Майбутні покращення:
  \begin{itemize}
    \item можливість детектувати дошку, що дасть приріст у швидкості обробки
          всього відео, оскільки панорама буде створюватись по кадрам меншого розміру;
    \item можливість детектувати секції дошки, яка розділена на декілька малих дошок;
    \item можливість отримати панорамне відео, а не слайди;
    \item можливість векторизувати написи на дошці;
    \item реалізувати програмну частину за допомогою більш ефективних засобів:
          наприклад, використовувати компільовану мову програмування
          та використовувати обчислення на GPU.
  \end{itemize}
\end{frame}
