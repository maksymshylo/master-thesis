\section{Методи локалізації людини та рухомих об'єктів}
\begin{frame}
  \frametitle{Алгоритм Бойкова Колмогорова}

  Сформулюємо задачу максимального потоку
  \begin{equation*}
    \sum_{t \in N_s} f_{st} \rightarrow \max_{f: \tau \rightarrow R }
  \end{equation*}
  з обмеженнями
  \begin{equation*}
    \begin{gathered}
      \begin{cases}
        f_{tt^{'}} \leq  c_{tt^{'}},                                   & \forall tt^{'}  \in \tau ,         \\
        \sum_{p \in P_t} f_{pt} - \sum_{t^{'} \in N_t} f_{tt^{'}} = 0, & \forall t \in T \setminus \{s,e\}, \\
        \sum f_{tt^{'}} \geq 0,                                        & \forall tt^{'}  \in \tau.
      \end{cases}
    \end{gathered}
  \end{equation*}
  Це означає, що
  \begin{enumerate}
    \item потік має не перевищувати пропускну здатність для всіх ребер;
    \item сума потоків, що входять у вузол не повинна змінитись на виході;
    \item потік завжди додатній.
  \end{enumerate}
\end{frame}
