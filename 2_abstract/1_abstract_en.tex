\chapter*{Abstract}

The thesis contains
\pageref{LastPage}
pages,
\TotalValue{totalfigures}
figures
and
\total{citenum} references.

Video recordings of lectures are no longer a rarity in
the conditions of distance learning.
Videos may be in an inconvenient format for
students or contain different artifacts due
to compression, camera quality, and other factors.
It is useful to have a presentation of the study material,
which contains only the text from the board
because such a view of the material is most
like the compendium.

The aim of this work
is creating an algorithm for obtaining panorama
slides without a teacher from video lecture
recorded by a camera which may move or shake.

To perform the study the following instruments were used:
\begin{itemize}
  \item
        SIFT for obtaining image's keypoints;
  \item
        homography for getting panorama slides;
  \item
        Boykov-Kolmogorov max-flow algorithm for creating mask of
        moving objects;
  \item
        convolutional neural networks for human detecting;
  \item
        denoising and binarizing operators.
\end{itemize}

Keywords:
\MakeUppercase{images processing, intellectual video
  processing, video stabilization}
