\chapter*{Реферат}

Дисертація містить
\pageref{LastPage}
сторінки,
\TotalValue{totalfigures}
ілюстрацій
і бібліографію з
\total{citenum}
найменувань.

Відеозаписи лекцій вже не рідкість в умовах дистанційного навчання.
Відео може бути у незручному форматі для студентів або містити різні артефакти
через стиснення із втратами, якість зйомки та  інші фактори. Добре було б мати
презентацію навчального матеріалу, яка містить лише текст з дошки,
оскільки це більш наближено до формату конспекту.

Метою роботи є розробка алгоритму,
який створює панорамні слайди без викладача з відео лекції
за умови, що камера може рухатись.

Для досягнення мети було використано:
\begin{itemize}
      \item SIFT алгоритм для знаходження ключових точок зображення;
      \item гомографію для отримання панорамних слайдів;
      \item алгоритм максимального потоку Бойкова-Колмогорова
            для отримання маски рухомих об'єктів;
      \item згорткові нейронні мережі для детекції людини;
      \item оператори знешумлення та бінаризації зображень.
\end{itemize}

Ключові слова:
\MakeUppercase{оброблення зображень, інтелектуальне оброблення відео,
      стабілізація відео}
