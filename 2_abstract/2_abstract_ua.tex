\chapter*{Реферат}

Дисертація містить 
%\pageref{LastPage} 
сторінки,
%\TotalValue{totalfigures} 
ілюстрацій
і бібліографію з 
%\total{citenum} 
найменувань.

Відеозаписи лекцій вже не рідкість 
в умовах дистанційного навчання.
Відео може бути у незручному форматі для 
студентів або містити різні артефакти 
через стиснення з втратами, якість зйомки та 
інші фактори. Добре було б мати презентацію навчального 
матеріалу, яка містить лише текст з дошки, 
оскільки це більш наближено до формату конспекту.

Метою роботи є 
створення алгоритму, який з відео лекції 
робить панорамні слайди без викладача.

Для досягнення мети було використано
\begin{itemize}
  \item
    Метод SIFT для знаходження ключових точок зображення;  
  \item
    Гомографію для отримання панорамних сладів;
  \item
    Алгоритм максимального потоку Бойкова-Колмогорова
    для отримання маски рухомих об'єктів;
  \item
    Оператори знешумлення та бінаризації зображень.
\end{itemize}

\MakeUppercase{оброблення зображень, інтелектуальне оброблення відео,
стабілізація відео}
