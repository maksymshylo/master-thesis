В результаті виконання роботи вдалося
розробити алгоритм створення панорамних слайдів без викладача.


Жодна попередня робота не мають всі можливості даної, що і стало однією 
з причин дослідження за даною темою.


Оглянуто теоретичне підґрунтя алгоритму Бойкова-Колмогорова та задачу
максимального потоку (мінімального розрізу), яку він вирішує. Виявлено, що 
даний метод досить добре справляється у задачі видалення рухомих об'єктів.
Також описано згорткові нейромережі сімейства YOLO, MobileNet, R-CNN та 
Faster-RCNN. Результати свідчать про високу якість детекції людини та можливість
їх використання на смартфонах. За експериментальними результатами, YOLOv5n є найбільш оптимальним
методом прибирання викладача.  


Було реалізовано демонстративне програмне забезпечення,
що приймає на вхід відео з параметрами від користувача та 
будує панорамні оброблені слайди. 


Якість результатів слайдів свідчить про проведення подальшої роботи та 
ще покращення як алгоритму так й інформаційної технології.
\clearpage