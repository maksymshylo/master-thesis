В результаті виконання роботи вдалося
розробити алгоритм створення панорамних слайдів без викладача.


Жоден із розглянутих аналогів не має всі можливості даної роботи, що і стало однією
з причин дослідження даної теми.


Оглянуто теоретичне підґрунтя алгоритму Бойкова-Колмогорова та задачу
максимального потоку (мінімального розрізу), яку він вирішує. Виявлено, що
даний метод досить добре справляється у задачі видалення рухомих об'єктів.
Результати роботи згорткових нейромереж сімейства  YOLO, MobileNet, SSD та
R-CNN свідчать про високу якість детекції людини та можливість
їх використання на смартфонах. За експериментальними результатами YOLOv5n є найбільш оптимальним
методом прибирання викладача.


Було реалізовано програмне забезпечення,
що приймає на вхід відео з параметрами від користувача та
будує панорамні оброблені слайди.


Якість результатів слайдів свідчить про необхідність проведення подальшої роботи та
ще покращення алгоритму й інформаційної технології. Наприклад детекція дошки прискорить
роботу системи, оскільки розмір панорами зменшиться і потрібно буде менше часу для її обробки.
Також в майбутньому стане в нагоді можливість отримання не растрового зображення написів дошки, а
векторного. Можливість створення панорамних слайдів із використанням графічного процесора також 
в рази прискорить роботу системи. 
\clearpage