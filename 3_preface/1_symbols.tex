\chapter*{Перелік умовних позначень, символів, одиниць, скорочень і термінів}
\addcontentsline{toc}{chapter}{Перелік умовних позначень, символів, одиниць, скорочень і термінів}

\textbf{Стандартні позначення}

 \noindent$\mathbb{N}$~---~множина натуральних чисел; \\
 \noindent$\mathbb{R}$~---~множина дійсних чисел; \\
 \noindent$\mathbb{R}^n$~---~$n$-вимірний векторний простір дійсних чисел; \\
\\
\\
\textbf{Позначення, введені в дисертації}

 \noindent$F$~---~множина кадрів відео; \\
 \noindent$T$~---~довжина відео (кількість кадрів); \\
 \noindent$P$~---~множина координат пікселів зображення; \\
 \noindent$C$~---~множина інтенсивностей пікселів зображення; \\
 \noindent$F_p^i$~---~інтенсивність пікселя з координатою $p$$\in$$P$; \\
 Б-К алгоритм ~---~алгоритм Бойкова-Колмогорова; \\
