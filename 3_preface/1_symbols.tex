\chapter*{Перелік умовних позначень, символів, одиниць, скорочень і термінів}
\addcontentsline{toc}{chapter}{Перелік умовних позначень, символів, одиниць, скорочень і термінів}

\textbf{Стандартні позначення}

 \noindent$\mathbb{N}$~---~множина натуральних чисел; \\
 \noindent$\mathbb{R}$~---~множина дійсних чисел; \\
%  \noindent$\mathbb{R}^n$~---~$n$-вимірний векторний простір над полем дійсних чисел; \\
%  \noindent$\mathbb{R}_+^n$~---~множина векторів із невід'ємними координатами в $\mathbb{R}^n$; \\
%  \noindent$\Delta^n$~---~$n$-вимірний симплекс; \\
%  \noindent$\Delta^X$~---~$\left| X \right|$-вимірний симплекс, координати якого проіндексовані елементами з множини $X$; \\
%  \noindent$\left| X \right| $~---~потужність множини $X$, або кардинальне число множини $X$; \\
%  \noindent$conv \left( X \right)$~---~опукла оболонка множини $X$; \\
%  \noindent$X \times Y$~---~декартів добуток множин $X$ та $Y$; \\
%  \noindent$\langle \pmb{x}, \pmb{y} \rangle$~---~скалярний добуток векторів $\pmb{x}$ та $\pmb{y}$; \\
%  \noindent$\mathcal{O} \left( \cdot \right)$~---~<<О>> велике, нотація Ландау;
\\
\\
\textbf{Позначення, введені в дисертації}

 \noindent$F$~---~множина кадрів відео; \\
 \noindent$T$~---~довжина відео (кількість кадрів); \\
 \noindent$P$~---~множина координат пікселів зображення; \\
 \noindent$C$~---~множина інтенсивностей пікселів зображення; \\
 \noindent$F_p^i$~---~інтенсивність пікселя з координатою $p$$\in$$P$; \\
 %\noindent$\mathcal{N} \left( v \right)$~---~множина вершин графу, інцидентних вершині $v$; \\
 %\noindent$\mathcal{N}$~---~множина пар інцидентних вершин графу;