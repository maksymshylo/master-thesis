\chapter*{Вступ}
\addcontentsline{toc}{chapter}{Вступ}

Дана робота завдячує Кригіну Валерію Михайлович~---~молодшому
науковому співробітнику Відділу обробки та розпізнавання
образів Міжнародного науково-навчального центру інформаційних
технологій і систем НАН України та МОН України.

\textbf{Актуальність роботи.}
Задача створення конспекту лекцій не є новою,
а з масовим переходом навчання у віддалений режим
стала більш актуальною проблема створення якісних
лекційних матеріалів, зокрема слайдів, які містять
стислу суть занять. Було б добре мати технологію, що 
перетворюватиме відео лекцію у стисле слайд шоу, на 
якому будуть написи з дшошки без викладача і до того 
ж всю область дошки, навіть якщо камера
рухалась.

\textbf{Мета і завдання дослідження.}

\textit{Об'єкт дослідження}~---~математичні методи 
реєстрації зображень.

\textit{Предмет дослідження}~---~автоматична обробка 
відеозаписів.

Метою роботи є розробка алгоритму, що перетворює відео 
лекції у панорамні знімки без викладача.

Завдання наступні:
\begin{enumerate}
  \item
    вивчити методи обробки та порівняння зображень, 
    які знадобляться для створення панорамних знімків;
    \item
    вивчити та доповнити математичні методи, 
    що допоможуть визначати рухомі об'єкти;
  \item
  розробити демонстраційне програмне забезпечення.
\end{enumerate}

\textbf{Методи дослідження:}
\begin{enumerate}
    \item опрацювання літератури за темою;
    \item створення теоретичного підгрунтя алгоритму;
    \item написання програми;
    \item аналіз отриманих результатів.
\end{enumerate}

\textbf{Наукова новизна одержаних результатів.}

Створено алгоритм та реалізувано інформаційну технологію 
для одержання панорамних слайдів з відео лекції.

\textbf{Практичне значення одержаних результатів.}

За допомогою програми викладач може надати короткий вміст 
відео матеріалу. Наступними кроками є автоматичне детектування 
дошки, розбиття її сектори та розпізнавання написаного на 
дошці тексту та формул.

\textbf{Публікації.}

.... Всеукраїнська науково-практична конференція студентів,
аспірантів та молодих вчених <<Теоретичні і прикладні проблеми фізики,
математики та інформатики>>.