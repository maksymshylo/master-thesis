\chapter*{Вступ}
\addcontentsline{toc}{chapter}{Вступ}

\textbf{Актуальність роботи.}
Задача створення конспекту лекцій не є новою
та стала ще більш актуальною,
оскільки з масовим переходом навчання у віддалений режим навчання
постала проблема створення якісних
лекційних матеріалів, зокрема слайдів, які містять
стислу суть занять. Корисно мати технологію, що
перетворює відео лекцію у стисле слайд шоу, на
якому будуть написи з дошки без викладача і до того
ж всю область дошки, навіть якщо камера
рухалась. 

Дана інформаційна технологія повинна мати ряд властивостей:
\begin{itemize}
      \item здатність працювати на звичайному смартфоні у режимі реального часу;
      \item можливість працювати з дошками різного кольору;
      \item можливість працювати з рухливою камерою;
      \item мінімальна кількість дефектів на слайдах,
        таких як наявність фрагментів викладача
        або видимі шви у місцях склейки кадрів.
\end{itemize}
На поточний момент жоден аналог не може такого дати, тому була поставлена
задача створити таку технологію.

\textbf{Мета і завдання дослідження.}

\textit{Об'єкт дослідження}~---~відеозаписи лекцій.

\textit{Предмет дослідження}~---~автоматична обробка
відеозаписів.

\textit{Метою роботи} є розробка алгоритму, що перетворює 
відеозапис лекції у панорамні знімки без викладача
за умови, що камера може рухатись.

Завдання наступні:
\begin{enumerate}
      \item
            вивчити та доповнити математичні методи,
            що допоможуть визначати рухомі об'єкти;
      \item
            опрацювати літературу про згорткові нейронні мережі, які застосовуються
            для знаходження в кадрі людини;
      \item
            вивчити методи обробки та порівняння зображень для створення панорамних знімків;
      \item
            розробити алгоритм перетворення відеозапису лекції у панорамні знімки;
      \item
            розробити демонстраційне програмне забезпечення.
\end{enumerate}

\textbf{Методи дослідження:}
\begin{enumerate}
      \item опрацювання літератури за темою;
      \item створення теоретичного підґрунтя алгоритму;
      \item написання програми;
      \item аналіз отриманих результатів.
\end{enumerate}

\textbf{Наукова новизна одержаних результатів.}

Створено алгоритм та реалізовано інформаційну технологію
для одержання панорамних слайдів з відеозапису лекції,
який створено за допомогою рухливої камери,
а дошку час від часу перекриває викладач та інші люди.

\textbf{Практичне значення одержаних результатів.}

За допомогою програми викладач може надати короткий вміст
відео матеріалу. Наступними кроками є автоматичне детектування
дошки, розбиття її на сектори та розпізнавання написаного на
дошці тексту та формул.

\textbf{Публікації.}
Стаття <<Перетворення відеозапису з дошки у слайд-шоу>> авторів 
Кригін В. М. та Шило М. К. прийнята до друку 
Міжнародним науковим журналом “Control Systems and Computers”
(“Системи керування та комп'ютери”) та буде опублікована в № 2 (298) 2022 року.
