\chapterConclusion

Для створення кінцевого алгоритму перетворення відео з дошки на слайди 
потрібно вивчити методи локалізації людини або рухливих об'єктів,
зробити стабілізацію відео та панорамне склеювання кадрів.


Алгоритм Бойкова-Колмогорова має та може бути застосованим для знаходження
рухомих об'єктів.
Виявлено, що згорткові мережі сімейств YOLO, MobileNet, SSD та R-CNN 
підходять для створення інформаційної технології побудови слайдів на
мобільних пристроях та можуть бути використані як методи прибирання викладача.


Експериментально доведено, що алгоритм знаходження ключових точок SIFT можна використовувати 
в задачі знаходження ключових точок на дошці. Також побудований кінцевий алгоритм створення
панорамного знімку без викладача (алгоритм \ref{al:panorama_creating_algorithm}) був запрограмований та 
підтвердив точність математичних формул. Д


Запропоновано використання швидкої медіани для знешумлення панорамних знімків, 
яка має складність $\mathcal{O}(h \cdot w)$ на відміну від звичайної темпоральної, в
якій має $\mathcal{O}(h \cdot w \cdot m^2)$. Дана заміна прискорила роботу інформаційної
технології, як і було доведено математично.
