\chapterConclusion

Зроблена постановка задачі створення слайдів. Описано ввідні
дані алгоритму та запропонована його блок-схема.
Таким чином, для створення інформаційної технології
потрібно вивчити методи локалізації людини або рухливих об'єктів та
зробити стабілізацію відео та панорамне склеювання кадрів.


Було розглянуто алгоритм Бойкова-Колмогорова для
отримання маски рухомих об'єктів. Також описано архітектури
згорткових мереж YOLO, MobileNet, SSD, R-CNN, їхні переваги та недоліки.
Виявлено, що дані методи підходять для
створення інформаційної технології побудови слайдів на
мобільних пристроях.


Було розглянуто алгоритм знаходження ключових точок SIFT.
Показано, як знаходити найкращу матрицю гомографії за
допомогою RANSAC. Побудовано кінцевий алгоритм створення
панорамного знімку без викладача (алгоритм \ref{al:panorama_creating_algorithm}).


Запропоновано метод прийняття рішення щодо того, чи обирати панорамний знімок як слайд, чи ні.
Оглянуто теоретичне підґрунтя темпоральної медіани для
знешумлення так покращення якості панорамних знімків.