\chapterConclusion

Для створення кінцевого алгоритму перетворення відео з дошки на слайди 
потрібно вивчити методи локалізації людини або рухливих об'єктів,
зробити стабілізацію відео та панорамне склеювання кадрів.


Алгоритм Бойкова-Колмогорова має та може бути застосованим для знаходження
рухомих об'єктів.
Виявлено, що згорткові мережі сімейств YOLO, MobileNet, SSD та R-CNN 
підходять для створення інформаційної технології побудови слайдів на
мобільних пристроях та можуть бути використані як методи прибирання викладача.


Емпіричним шляхом дійшли висновку, що алгоритм знаходження ключових точок SIFT можна використовувати 
в задачі знаходження ключових точок на дошці. Також побудований кінцевий алгоритм створення
панорамного знімку без викладача (алгоритм \ref{al:panorama_creating_algorithm}) був запрограмований та 
підтвердив точність математичних формул.


Запропоновано використання швидкої медіани для знешумлення панорамних знімків, 
зі складністю $\mathcal{O}(h \cdot w)$ на відміну від звичайної темпоральної,
яка має $\mathcal{O}(h \cdot w \cdot m^2)$.
