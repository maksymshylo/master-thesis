\section{Постановка задачі}

Головними цілями даної роботи створити власну інформаційну технологію
на вхід якої буде даватись відео-лекція, а на вихід панорамні слайди
без викладача.

Для одержання бажаних результатів було запропоновано
вирішити такі проблеми:
\begin{enumerate}
	\item Стабілізація кадрів;
	      Багато відео лекцій записуються не в найкращих умовах.
	      Камера може випадково затруситися, якщо її прикріпити до столу
	      де студенти пишуть лекцію f,j або закріплена не стабільно, таким чином
	      це заважає нормально слухати лекцію і фокусуватись на написах дошки.
	      
	      *****Вставити  два кадри і різницю, щоб показати не стабілізоване відео
	      
	\item Прибирання викладача;
	      Також однією з головних задач є видалення викладача з відео, оскільки
	      було поставлено за мету надавати користувачу чисті написи з дошки.
	      Є чимало методів як вирішувати цю задачу. Можна локалізувати тільки
	      людину, а можна і всі об'єкти, що рухаються в площині дошки. Маючи
	      дані про положення в кадрі викладача ми матимемо змогу оновлювати
	      з часом інформацію з дошки яка була тимчасово перекрита.
	      
	      *****Вставити кадр з маскою людини
	      
	\item Створення панорамних слайдів;
	      Часто, коли дошки сильно об'ємні по площі камера знімає тільки ту частину де
	      викладач щось пише. Під час лекції камеру переміщають так, щоб лектор був завжди
	      в полі зору. Таким чином, якщо студент не встиг переписати все з однієї частини
	      дошки, а камеру перемістили, то учень втрачає частину інформації. Було б добре
	      мати слайди, які по мірі руху камери містили всю дошку.
	      
	      *****Вставити панорамну склейку кадрів
	      
	\item Оцифровування написів;
	      Умови освітлення, розводи на дошці, неякісна зйомка, шуми - все це сильно впливає
	      на сприйняття лекційного матеріалу. Непогано було б мати бінаризовані слайди без
	      шумів тільки з написами як в записнику.
	      
\end{enumerate}


