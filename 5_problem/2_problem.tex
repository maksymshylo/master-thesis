\section{Постановка задачі}

Головна ціль даної роботи ~---~ це створення власної інформаційної технології,
на вхід якої буде подаватись відеозапис лекції, а на вихід панорамні слайди
без викладача.
Для одержання бажаних результатів було запропоновано
вирішити такі проблеми:
\begin{enumerate}
	\item
	      Стабілізація кадрів.
	      Багато відео лекцій записуються не в найкращих умовах.
	      Камера може випадково затруситися, якщо її прикріпити до столу,
	      де студенти пишуть лекцію, або може бути нестабільно закріплена.
	      Трясіння камери може заважати сприймати лекцію і фокусуватись на написах дошки.
	\item
	      Прибирання викладача.
	      Існує чимало методів вирішення цієї задачі. Можна локалізувати тільки
	      людину, а можна і всі об'єкти, що рухаються в площині дошки. Маючи
	      дані про положення викладача в кадрі, ми матимемо змогу оновлювати
	      з часом інформацію з дошки, яка була тимчасово перекрита.

	\item
	      Створення панорамних слайдів.
	      Часто, коли дошки мають значну площу, камера знімає тільки ту частину, де
	      викладач щось пише. Під час лекції камеру переміщують так, щоб лектор був завжди
	      в полі зору. Таким чином, якщо студент не встиг переписати все з однієї частини
	      дошки, а камеру перемістили, втрачається частина інформації. Тому корисно
	      мати слайди, які по мірі руху камери містять всю дошку.

	\item
	      Оцифровування написів.
	      Умови освітлення, розводи на дошці, неякісна зйомка, шуми ~---~ все це впливає
	      на сприйняття лекційного матеріалу. Корисно мати бінаризовані слайди,
	      які містять лише написи як в записнику.
\end{enumerate}
