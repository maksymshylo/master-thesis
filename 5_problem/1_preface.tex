У другому розділі описується постановка головних 
задач та можливих рішень для отримання панорамних знімків.
Описуються методи визначення області,
де знаходиться людина або рухомі об'єкти,
щоб у подальшому прибрати викладача з панорамних знімків.
Розглядається будова алгоритму Бойкова-Колмогорова та згорткових мереж 
YOLO, MobileNet, SSD, R-СNN. Описується пошук відповідності між кадрами,
матриця гомографії та створення панорамного
знімку без викладача.
Описується процес прийняття рішення щодо того,
чи брати панорамний знімок як слайд, чи ні. 
Також наводиться приклад роботи оператору Лапласа та робиться огляд швидкої медіани для
зменшення шуму.