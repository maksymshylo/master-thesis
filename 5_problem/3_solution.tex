\section{Пропоновані рішення}


У даній роботі описується алгоритм напівавтоматичного
створення слайдів на основі відеозапису лекції (Рис. \ref{fig:pipeline}).
За допомогою наведеного алгоритму можна обробляти відео
у режимі реального часу, адже кожна його ітерація потребує
лише поточний на попередні кадри. Між кожною парою
сусідніх кадрів відео розраховується бінарна маска областей, де
потенційно знаходяться рухомі об'єкти або викладач в залежності від методу
створення маски.

Ті частини зображення, де не було помічено рух або людина,
зберігаються до зображення, яке ми в даній роботі називаємо панорамою, тому що
це зображення може розширюватись у випадку, коли камера рухається
навмисно або хитається через небажаний вплив на неї.
Отриману панораму ми порівнюємо з тими панорамами, що були отримані на
попередніх кроках, і створюємо новий слайд, якщо було помічено
достатню кількість змін.

Інформаційна технологія, що реалізує даний алгоритм, працює у
напівавтоматичному режимі ~---~ вона потребує від користувача введення
деяких параметрів, а саме:
\begin{enumerate}
    \item крок або кількість кадрів між тими, які беруться в обробку;
    \item ступінь згладжування маски рухомих об'єктів для алгоритму Бойкова-Колмогорова;
    \item рівень довіри присутності людини для згорткових мереж;
    \item крок між панорамами для перевірки необхідності створення нового панорамного слайду;
    \item кількість необхідних змін між двома панорамами для створення нового слайду.
\end{enumerate}

\usetikzlibrary{arrows,positioning,shapes}
\begin{figure}[H]
    \begin{center}
        \begin{tikzpicture}[node distance=4mm, >=latex',
                block/.style = {draw, rectangle, minimum height=10mm, minimum width=28mm,align=center},
                rblock/.style = {draw, rectangle, rounded corners=0.5em},
                tblock/.style = {draw, trapezium, minimum height=10mm,
                        trapezium left angle=75, trapezium right angle=105, align=center},
            ]
            \node [rblock]                           (video)        {Відео};
            \node [block, below=of video]            (mov_objects)  {Побудова маски\\
                рухомих об’єктів або людини};
            \node [block, right=of mov_objects]      (pan)          {Побудова панорамами};
            \node [block, right=of pan]              (denoise)      {Зниження рівня\\ шуму};
            \node [rblock,above=of denoise]          (slides)       {Слайди};

            \path[draw,->] (video)         edge    (mov_objects)
            (mov_objects)   edge    (pan)
            (pan)           edge    (denoise)
            (denoise)       edge    (slides)
            ;
        \end{tikzpicture}
    \end{center}
    \caption{Процедура створення панорамних слайдів
        \label{fig:pipeline}
    }
\end{figure}
\section{Вхід алгоритму}
Для опису алгоритму створення панорамних слайдів потрібно визначити та
описати об'єкти з якими взаємодіє інформаційна технологія.

Дошка ~----~ це плоска поверхня, яку знімає камера. Це може бути крейдяна
дошка, маркерна дошка, стіна тощо.

Записи ~---~ це ті місця дошки, де відбулася зміна кольору, яка тривала
відносно довгий час. Важливо зауважити, що ці записи мають бути
саме у площині дошки, при чому людина, яка проходить біля неї, не вважається
зміною кольору, оскільки цей рух був не тривалим.

В область огляду камери має потрапляти дошка або її частина. Наведений
алгоритм не розрахований на відео, що містить декілька дошок, які знаходяться
не в одній плоскій площині. Камера, що знімає це відео, може рухатись, проте
чим більше вона нерухома ~---~ тим краще (алгоритм не буде працювати, якщо камера
рухається постійно). Дошка на відео може перекриватися сторонніми об'єктами,
проте бажано, щоб ці об'єкти були рухливими, щоб алгоритм виявлення рухомих
об'єктів їх помічав та для подальшого оновлення написів на дошці.

Кадром під номером $i$ шириною $w$ і висотою $h$ називаємо відображення
$P \to C$, де $P$ ~---~ множина координат пікселів, $С$ ~---~ скінченна множина можливих рівнів
яскравостей (інтенсивностей) пікселів.
\begin{equation*}
    P = [1, \ldots ,w]\times[1, \ldots, h], C \subset R
\end{equation*}
Відео \(F\) довжиною \(T\)
є послідовністю \(\left( F^{i}:i = \overline{1,T} \right)\) кадрів
\(F^{i}:P \rightarrow C\). Той факт, що піксель з координатою
\(p \in P\) на кадрі \(F^{i}\) має інтенсивність \(c \in C\),
позначатимемо \(F_{p}^{i} = c\).

\clearpage