\subsection{Вхід алгоритму}

Для опису алгоритму створення панорамних слайдів потрібно визначити 
вхідні дані та параметри, які буде надавать користувач інформаційної
технології.


Дошка - це плоска поверхня, яку знімає камера. Це може бути крейдяна 
дошка, маркерна дошка, стіна тощо.

Записи - це ті місця дошки, де відбулася зміна кольору, яка тривала 
відносно довгий час. Важливо зауважити, що ці записи мають бути 
саме у площині дошки, при чому людина яка проходить біля неї не вважається
зміною кольору, оскільки цей рух був не тривалим.

В область огляду камери має потрапляти дошка або її частина. Наведений 
алгоритм не розрахований на відео, що містить декілька дошок, які знаходяться
не в одній плоскій площині. Камера, що знімає це відео, може рухатись, проте 
чим більше вона нерухома, тим краще: алгоритм не буде працювати, якщо камера 
рухається постійно. Дошка на відео може перекриватися сторонніми об’єктами, 
проте бажано, щоб ці об’єкти були рухливими, щоб алгоритм виявлення рухомих 
об’єктів їх помічав.

Кадром під номером $i$ шириною $w$ і висотою $h$ називаємо відображення, де 
$P$$\to$$C$, де $P$ - множина координат пікселів, $С$ - скінченна множина можливих рівнів
яскравостей (інтенсивностей) пікселів.
\begin{equation}
    P = [1, \ldots ,w]\times[1, \ldots, h], C \subset R
\end{equation},

Відео \(F\) довжиною \(T\)
є послідовністю \(\left( F^{i}:i = \overline{1,T} \right)\) кадрів
\(F^{i}:P \rightarrow C\). Той факт, що піксель з координатою
\(p \in P\) на кадрі \(F^{i}\) має інтенсивність \(c \in C\),
позначатимемо \(F_{p}^{i} = c\).


\subsection{Побудова маски рухомих об'єктів}

Під час лекцій часто виникають такі ситуації, коли викладач затуляє
собою частину дошки з написами -- наприклад, для запису нового матеріалу
або щоб видалити старі написи з дошки. Іноді студенти просять викладача
відійти від дошки, щоб переписати з неї те, що з'явилося на ній лише
хвилину тому, проте наша програма дозволяє побачити частину дошки, яку
затулив викладач, якщо перед цим перекритий сегмент було добре видно
протягом вказаного користувачем часу.

Якщо вирішити проблему прибирання викладача з відео, то студент завжди 
буде бачити, що відбувається на дошці.

Для вирішення цієї задачі ми будуємо маску рухомих об'єктів,
щоб потім їх видаляти. Рухомими об'єктами можуть бути викладач, 
студенти, а також записи, що тільки-но з'явились.

Було запропоновано декілька методів створення маски викладача чи 
рухомих об'єктів.

\subsection{Алгоритм Бойкова-Колмогорова}

Даний алгоритм є вирішенням задачі знаходження мінімального розрізу графу
або еквівалентної їй максимального потоку. 

Ведемо позначення:
% $T$ - множина вузлів, 
% \tau - множина направлених дуг,
% $s$ - джерело (початок),
% $e$ - стік (кінець),
% $N_t = {t^{'}: tt^{'} \in \tau }$,
% $P_t = {t^{'}: t^{'}t \in \tau }$,
% $f$ - потік,
% $c$ - пропускна здатність,

Задача максимального потоку

\begin{equation}
    \sum_{t \in N_s} f_{st} \rightarrow \max_{f: \tau \rightarrow R }
\end{equation}

З обмеженнями: 
\begin{equation} 
    \begin{gathered}
        \begin{cases}
            f_{tt^{'}} \ll c_{tt^{'}}, &  \forall tt^{'}  \in \tau , \\

            \sum_{p \in P_t} f_{pt} - \sum_{t^{'} \in N_t} f_{tt^{'}} = 0, & 
            \forall t \in T \setminus \{s,e\}, \\

            \sum f_{tt^{'}} \geq 0, & \forall tt^{'}  \in \tau
        \end{cases}
\end{gathered}
\end{equation}

Що значать:
\begin{enumerate}
    \item 1. Потік має не перевищувати пропускну здатність для всіх ребер;
    \item 2. Сума потоків, що входять у вузол не повинна змінитись на виході;
    \item 3. Потік завжди додатній.
\end{enumerate}

Оригінальне рішення 

\begin{algorithm}
    \caption{Алгоритм максимального потоку}
    \begin{algorithmic}
    \State \textbf{Вхід} Граф, з об'єктами $t \in T$, ребрами $tt^{'}$;
    \State \textbf{Вихід} $F_{maxflow}$ - значення максимального потоку;
    \State \textbf{Ініціалізація}: $F_{maxflow} = 0; f_{tt^{'}}^{0} = 0 \forall tt^{'}  \in \tau$.
    \State \textbf{Поки існує шлях з $s$ в $e$}:    
    \State \textbf{Крок 1}: Знаходимо шлях від $s$ до $e$.
    (алгоритм Едмонса-Карпа або Форда Фалкерсона).
    \State Відвідуємо $t^{'}$ із $t$ якщо:
            \State \qquad 1. $f_{tt^{'}} \neq   c_{tt^{'}}$;
            \State \qquad 2. $ \nexists p_{t^{'}}^{i} \Rightarrow  p_{t^{'}}^{i} = t $ (запам'ятали вершину);
            \State \qquad 3. $ t^{'} \neq s $
    \State \textbf{Крок 2}: Проходимо по знаданому шляху: 
            \State \qquad 1. Знаходимо $ \vartriangle f^{i} = \min_{tt^{'} \in \{шлях із s в e \}} $;
            \State \qquad 2. Змінюємо потік: $ f_{tt^{'}}^{i+1} = f_{tt^{'}}^{i} + \vartriangle f^{i} $;
            \State \qquad 3. Оновлюємо $F_{maxflow}$: $ F_{maxflow}^{i+1}  = F_{maxflow}^{i} + \vartriangle f^{i} $ ;
    \end{algorithmic}
\end{algorithm}

Знайшовши максимальний потік можемо знайти мінімальний зріз:
Для $\forall t \in T$ знайти $\theta_{t} \in \{0,1\}$
\begin{algorithm}
    \caption{Алгоритм пошуку розмітки}
    \begin{algorithmic}
    \State \textbf{Вхід}:Граф з насиченими дугами після знаходження максимального потоку.
    \State \textbf{Вихід}: $\theta$ - розмітка графа.
    \State Запускаємо пошук в ширину або глибину вже з оновленим графом.
    \State Поки існує шлях з $s$ в $e$:
    \State Проходимось по всім об'єктам $ t \in T $:
    \State \qquad Якщо $c_{tt^{'}} \geqslant f_{tt^{'}}$, і $c_{tt^{'}} \geqslant 0$
    $\Rightarrow \theta_{t^{'}} = 1 $,
    \State \qquad інакше $\theta_{t^{'}} = 0 $

    \end{algorithmic}
\end{algorithm}