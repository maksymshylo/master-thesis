\subsection*{Вхід алгоритму}

Для опису алгоритму створення панорамних слайдів потрібно визначити 
вхідні дані та параметри, які буде надавать користувач інформаційної
технології.


Дошка - це плоска поверхня, яку знімає камера. Це може бути крейдяна 
дошка, маркерна дошка, стіна тощо.

Записи - це ті місця дошки, де відбулася зміна кольору, яка тривала 
відносно довгий час. Важливо зауважити, що ця записи мають бути 
саме у площині дошки, при чому людина яка проходить біля неї не вважається
зміною кольору, оскільки цей рух був не тривалим.

В область огляду камери має потрапляти дошка або її частина. Наведений 
алгоритм не розрахований на відео, що містить декілька дошок, які знаходяться
не в одній плоскій площині. Камера, що знімає це відео, може рухатись, проте 
чим більше вона нерухома, тим краще: алгоритм не буде працювати, якщо камера 
рухається постійно. Дошка на відео може перекриватися сторонніми об’єктами, 
проте бажано, щоб ці об’єкти були рухливими, щоб алгоритм виявлення рухомих 
об’єктів їх помічав.

Кадром під номером $i$ шириною $w$ і висотою $h$ називаємо відображення, де 
$P$$\to$$C$, де $P$ - множина координат пікселів, $С$ - скінченна множина можливих рівнів
яскравостей (інтенсивностей) пікселів.
\begin{equation}
    P = [1, \ldots ,w]\times[1, \ldots, h], C \subset R
\end{equation},


Відео
\begin{equation}
    F^i: i = \overline{1,T},  F^i: P \to C
\end{equation},